\documentclass{article}
\usepackage[utf8]{inputenc}
\usepackage[spanish]{babel}
\usepackage{listings}
\usepackage{graphicx}
\graphicspath{ {images/} }
\usepackage{cite}

\begin{document}

\begin{titlepage}
    \begin{center}
        \vspace*{1cm}
            
        \Huge
        \textbf{PRIMEROS PASOS PARA DESARROLLAR MI PROYECTO FINAL}
            
        \vspace{0.5cm}
        \LARGE
        {IDEA DE VIDEO JUEGO}
        \vspace{1.5cm}
            
        \textbf{Karen López Aljure}
            
        \vfill
            
        \vspace{0.8cm}
            
        \Large
        Despartamento de Ingeniería Electrónica y Telecomunicaciones\\
        Universidad de Antioquia\\
        Medellín\\
        Marzo de 2021
            
    \end{center}
\end{titlepage}

\tableofcontents
\newpage
\section{sección introductoria}\label{intro}
El siguiente trabajo consiste en plasmar la idea de mi primer video juego para mi proyecto final de laboratorio de informática II.

\section{Sección de contenido} \label{contenido}

\subsection{nombre del video juego}
GLUTTONOUS RIDDLE


\subsection{Caracteristicas}
%
sistema interactivo de múltiples adivinanzas y acertijos donde el jugador deberá encontrar la palabra oculta para ir pasando al siguiente nivel de dificultad del juego,el jugador puede escoger diferentes personajes,genero y grado de dificultad.

\subsection{Mercado}
%
va dirigido a edades desde los 7 años hasta los 18 años


\subsection{Sinopsis}
%
El juego inicia cuando el jugador ya haya escogido su personaje y grado de dificultad,en pantalla vera el acertijo que debe de adivinar para ir subiendo de nivel,aparecerán en pantalla algunas pistas  que lo ayudaran,tendrá 4 opciones de escribir la respuesta correcta;de no hacerlo perderá la opurtanidad de adivinarla y como castigo deberá comer unos alimentos que harán que el personaje vaya cambiando su figura físicamente, después de perder los 4 intentos su físico no soportara y explotara,adémas de perder.Por el contrario, si logra ganar podra continuar con el siguiente acertijo y continuar el juego . Cuando el jugador logre pasar cada uno de los niveles ganará puntos, vestuarios nuevos y accesorios.  

\subsection{Apariencia y Ambientación}
%
la plataforma del video juego tendrá una apariencia colorida y didáctica,los personajes se verán en un inicio de un tamaño pequeno y a medida que avancen en el juego irán cambiando su tamaño (aumentan en estatura y peso).

\subsection{Objetivo del juego}
%
superar los tres niveles adivinando el mayor número de acertijos con menos registro de intentos fallidos.

\subsection{Mecanica del juego}
%
para iniciar el juego el jugador debe escoger primero como va a ser su personaje,vestuario,género y nivel de dificultad del juego.el juego contará con tres niveles cuando el jugador inicie aparecera dentro del área de juego , se podrá visualizar al personaje ubicado encima de un mapa que muestra el rrecorido de cada uno de los niveles del juego desde el más fácil hasta el más complicado;cada nivel contará con diferentes números de acertijos.
primer nivel:tendrá 3 acertijos y contará con una pista que le servirá de ayuda al jugador para adivinarlo,el jugador contará con 4 intentos disponibles para lograr descifrar el acertijo,de no lograrlo aparecerán en pantalla varias figuras en forma de comida ,las cuales seranla forma de castigo por fallar en cada intento.el juegador deberá escoger solo uno de los alimentos que aparezca y este tendrá un efecto en el personaje , lo hará cambiar su figura un poco más inflada ,a medida que vaya fallando irá escogiendo y comiendo otro alimento diferente . al cabo de 4 intentos fallidos perderá y este estará muy inflado y explotara.Perdera,si logra pasar de acertijo ganara puntos y pasara al siguiente nivel.
segundo nivel: tendra 5 acertijos de mayor grado de dificultad que el primero y se jugara de igual forma que el anterior siendo que aca solo tendra 3 intentos por acertijo.
tercer nivel: tendra 7 acertijos más complicados , en este nivel tendrá por acertijo dos pistas que le ayudarán a adivinarlo y se jugará de la misma forma que los anteriores niveles.
cuando el jugador supere satisfactoriamente cada nivel obtendrá como premios vestuarios,accesorios y acumulará un puntaje ,el cual el sistema del juego irá registrando para guardar los mejores récords hechos por los jugadores y este los escalonara del 1 al 5 los mejores puestos.

\subsection{Definición del menú}
%
el menú será sencillo y fácil de usar,el jugador podrá observar en pantalla las opciones de elegir su personaje, vestuario, genero, nivel de dificultad.

\subsection{Información que se verá en pantalla}
%
botón de inicio de juego
botón de pausa
botón de reanudar
cantidad de intentos de juego y intentos fallidos
mapa del juego con sus niveles
premios obtenidos
marcador d puntos
opciones de vestuario de personajes
género del personaje

\subsection{Controles}
%
el juego podrá jugarse desde un pc , con ayuda del teclado podrá ingresar las respuestas de los acertijos ,para desplazarse en el juego utilizara las flechas del teclado.

\subsection{Sonido}
%
cada nivel contará con musicalización de intriga y suspenso.
Cuando pierda alguna oportunidad sonorá una alarma.
Cuando gane tendra un sonido variado de instrumentos musicales que demuestre esa alegría de haber tiunfado.

\subsection{Movimiento del personaje}
%
iniciando el juego el personaje podra moverse sobre el mapa que muestra la ubicación de los niveles del juego,verá los niveles de dificultad desde el más facil al más complicado,luego de saber que nivel jugar el personaje ingresará en ese nivel y empezará la diversión, cuando supere el nivel escogido inicialmente saldrá de nuevo en el mapa y escogerá el otro nivel y así mismo continuará con el ultimo nivel.

\section{segundo video juego}\label{intro}


\subsection{nombre del video juego}
Demobyrinth


\subsection{Caracteristicas}
%
El juego trata de un personaje que cae en un túnel, y debe de tratar de escapar de ahí, pero algo se lo impide un demonio que se alimenta de los humanos.

\subsection{Mercado}
%
va dirigido a edades desde los 7 años hasta los 18 años


\subsection{Sinopsis}
%
El personaje es un estudiante que iba caminado a su escuela, pero se cae en un túnel, el jugador debe de tratar de escapar de ahí, pero algo se lo impide un demonio que se alimenta de los humanos. La misión de el es buscar las llaves para abrir las puertas que encuentre y buscar la salida.
Cuando se abre una puerta la velocidad de la criatura aumenta, en el juego el personaje está en una tubería en forma de laberinto, haciéndole más difícil encontrar las llaves para lograr escapar de la criatura evitando que esta lo atrape.
Algunas de las llaves que el jugador ira encontrando le dará ciertas habilidades como ralentizar la criatura para que esta no logre atraparlo, se podrá topar con ciertos obstáculos como saltar ciertas partes para llegar al otro lado del camino, la criatura para saber si esta cerca hace un ruido o un palpito de corazón.

\subsection{Apariencia y Ambientación}
%
La plataforma del video juego tendrá una apariencia aterradora y con poca luz, serán laberintos y tendrá objetos luminosos.

\subsection{Objetivo del juego}
%
Lograr salir del túnel..


\subsection{Definición del menú}
%
el menú será sencillo y fácil de usar,el jugador podrá observar en pantalla las opciones de elegir su personaje, vestuario, genero, nivel de dificultad.

\subsection{Información que se verá en pantalla}
%
botón de inicio de juego
botón de pausa
botón de reanudar
mapa del juego 
premios obtenidos
marcador de puntos
opciones de vestuario de personajes
género del personaje

\subsection{Controles}
%
el juego podrá jugarse desde un pc , con ayuda del teclado podrá desplazarse en el juego utilizara las flechas del teclado y para saltar utilizara barra espaciadora.

\subsection{Sonido}
%
cada nivel contará con musicalización de intriga y suspenso.
sonidos de gritos de personas
grujido del mousntruo.

\subsection{Movimiento del personaje}
%
El personaje caerá dentro del túnel, el tendrá que correr del monstruo, utilizara saltos para avanzar más rápido en el juego.





\end{document}
